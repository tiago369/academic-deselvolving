\chapter{Dia: 10/02/22}
\label{chap:10-02-22}

Nesse dia foi majoritariamente focado no do desenvolvimento da programação para o funcionamento do Dobot.

A demo do DOBOT funciona utilizando Services e o \cite{Dobotand76:online} fez um tópico para poder movimentar o braço. Assim o primeiro desafio foi aprender a utilizar os Services do ROS, para isso foi utilizado a própria wiki oficial \cite{rospyOve3:online}.

\begin{lstlisting}[language=Python]
    #!/usr/bin/env python
    import rospy

    from dobot.srv import GetPose

    if __name__ == '__main__':
        print('1')
        # rospy.wait_for_service('print_pose')
        get_pose = rospy.ServiceProxy('DobotServer/GetPose', GetPose)
        print('2')


        try:
            print('a')
            print(get_pose())
            print('b')
        except rospy.ServiceException as exc:
            print("Service did not process request: " + str(exc))

        var = get_pose()
        print(var.x)
\end{lstlisting}

Dessa forma foi desenvolvido um código para poder exibir na tela a posição atual, já que seria necessário receber a posição do braço para poder fazer os cálculos da sua movimentação. Assim após finalizado foram incrementados no código position\underline{\space}control.py para controle de posição desenvolvido esse services

\begin{lstlisting}[language=Python]
    #!/usr/bin/env python
    from turtle import pu
    import rospy
    from geometry_msgs.msg import Pose
    import time
    from dobot.srv import GetPose
    import math

    def distancia(ini, fim):
        return math.sqrt((fim - math.sqrt((ini)**2))**2)

    def position():
        rospy.init_node('position_control', anonymous=True)
        publisher = rospy.Publisher('geometry_pose', Pose, queue_size=10)
        get_pose = rospy.ServiceProxy('DobotServer/GetPose', GetPose)


        print("Say where you want me to go")
        x = int(input("X axis: "))
        y = int(input("Y axis: "))
        z = int(input("Z axis: "))    # -*- coding: utf-8 -*-
    import rospy
    from geometry_msgs.msg import Pose

    # Vai importar todos os srv do dobot
    from dobot.srv import SetEndEffectorSuctionCup
    from dobot.srv import GetPose
    from dobot.srv import SetHOMEParams

    if __name__ == "__main__":
        rospy.init_node('pick_and_place', anonymous=True)
        suction_srv = rospy.ServiceProxy('DobotServer/SetEndEffectorSuctionCup', SetEndEffectorSuctionCup)
        get_pose = rospy.ServiceProxy('DobotServer/GetPose', GetPose)
        set_home = rospy.ServiceProxy('DobotServer/SetHOMEParams', SetHOMEParams)
        publisher = rospy.Publisher('geometry_pose', Pose, queue_size=10)
        # subscriber = rospy.Subscriber
        
        pose = get_pose()
        home = set_home(0, 0, 0, 0, False)
        ini_x = pose.x
        ini_y = pose.y
        ini_z = pose.z
        
        print('a')
        # Valores arbitrarios para delimitar a area de atuacão do robo
        while True:
            print('The actual pose is')
            print('X: ')
            print(pose.x)
            print('Y: ')
            print(pose.y)
            print('Z: ')
            print(pose.z)
            print()

            print("Say where the object is positioned in the table")
            x = int(input("X axis: "))
            y = int(input("Y axis: "))
            z = int(input("Z axis: "))
            if 0 <= x > 200:
                print('X value is not acceptable')
            elif -100 < y > 100:
                print('Y value is not acceptable')
            elif -100 < z > 100:
                print('Z value is not acceptable')
            else:
                break

        freq = rospy.Rate(10)

        while not rospy.is_shutdown():
            msg = Pose()
            resp = suction_srv(1, 0, False)

            print('Move ate o ponto para pegar o objeto')
            # dist = 1
            while (pose.x != x and pose.y != y and pose.z != z) or not rospy.is_shutdown():
                msg.position.x = x
                msg.position.y = y
                msg.position.z = z
                publisher.publish(msg)
                # dist = ((x - pose.x) + (y - pose.y) + (z - pose.z))/3

                

            print('Suga')
            resp = suction_srv(1, 254, False)

            # dist = 1
            print('Move ate o ponto inicial')
            while (pose.x != ini_x and pose.y != ini_y and pose.z != ini_z) or not rospy.is_shutdown():
                msg.position.x = ini_x
                msg.position.y = ini_y
                msg.position.z = ini_z
                publisher.publish(msg)
                dist = ((x - pose.x) + (y - pose.y) + (z - pose.z))/3

            print('Para de sugar')
            resp = suction_srv(1, 0, False)

            freq.sleep()

            break

        while dist >= 0.1:
            print('a')
            msg = Pose()
            pose = get_pose()
            
            msg.position.x = k * distancia(pose.x, x)
            msg.position.y = k * distancia(pose.y, y)
            msg.position.z = k * distancia(pose.z, z)
            publisher.publish(msg)

            dist = (distancia(pose.x, x) + distancia(pose.z, z) + distancia(pose.y, y)) / 3

            freq.sleep()

    if __name__ == "__main__":
        try:
            position()
        except rospy.ROSInterruptException:
            pass
\end{lstlisting}

Esse código foi feito baseado no código desenvolvido por \cite{Dobotand76:online}. Porém, nenhuma das suas versões conseguiram movimentar o braço robótico.

Depois de testar a movimentação decidi fazer um codigo para controlar as ferramentas do manipulador. Como a função do manipulador é pegar objetos as ferramentas escolhidas foram o \textit{gripper} e o \textit{suction cup}, ambos de funcionamento pneumático.

\begin{lstlisting}[language=Python]
    #! /usr/bin/env python
    from numpy import uint8
    import rospy
    from dobot.srv import SetEndEffectorSuctionCup

    if __name__ == "__main__":
        rospy.init_node('suction_test', anonymous=True)
        suction_srv = rospy.ServiceProxy('DobotServer/SetEndEffectorSuctionCup', SetEndEffectorSuctionCup)

        x = uint8(input('Put value:  '))

        freq = rospy.Rate(10)

        while not rospy.is_shutdown():
            resp = suction_srv(1, x, False)
            print(resp.result)
            resp.result
            freq.sleep()
\end{lstlisting}

O código acima foi desenvolvido para testar o service do \textit{suction cup}. Assim foi possível ativar a bomba pneumática e desativa-la. Porém a bomba não faz o movimento de succionar, apenas de soprar ar, mesmo testando diversas combinações com as variáveis do serviço.

\begin{lstlisting}[language=Python]
    from numpy import uint8
    import rospy
    from dobot.srv import SetEndEffectorGripper

    if __name__ == "__main__":
        rospy.init_node('suction_test', anonymous=True)
        suction_srv = rospy.ServiceProxy('DobotServer/SetEndEffectorGripper', SetEndEffectorGripper)

        x = uint8(input('Put value:  '))

        freq = rospy.Rate(10)

        while not rospy.is_shutdown():
            resp = suction_srv(1, x, False)
            print(resp.result)
            resp.result
            freq.sleep()
\end{lstlisting}

O código acima teve como objetivo testar o service do \textit{gripper}. Testando descobriu que este service também consegue ativar a bomba, mas não desativa-la nem sugar ar. Para que possa desativa-la foi usado o service do \textit{suction cup}. Dessa forma foi decidido utilizar o service do \textit{suction cup} chamado de \textit{SetEndEffectorSuctionCup} para ambos, ja que só precisa ativar e desativar a bomba para poder usar ambos.

Alguns problemas mecanicos também foram encontrados a parte que parafusaria o \textit{gripper} no braço fica muito folgada, dessa forma ele não prende no braço, mas tem como utilizar essa ferramenta se fizer um calço ou adaptador para prende-lo. Por outro lado o \textit{suction cup} consegue ser preso facilmente no manipulador, ela não tem como segurar objetos pois a bomba não faz a sucção
