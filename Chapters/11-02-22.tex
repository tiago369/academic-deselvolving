\chapter{Dia: 11/02/22}
\label{chap:11-02-22}

Foi corrigido o código do pick\underline{\space}and\underline{\space}place.py o DOBOT conseguiu se mover como mostrado no video seguinte: \href{https://youtu.be/nkaut31VhHA}{Dobot} 
Porém a bomba ainda continua soprando ar ao invés de puxar e o braço ainda continua com o movimento muito lento

\begin{lstlisting}
    #!/usr/bin/env python
    # -*- coding: utf-8 -*-
    import rospy
    from geometry_msgs.msg import Pose

    # Vai importar todos os srv do dobot
    from dobot.srv import SetEndEffectorSuctionCup
    from dobot.srv import GetPose
    from dobot.srv import SetHOMEParams

    if __name__ == "__main__":
        rospy.init_node('pick_and_place', anonymous=True)
        suction_srv = rospy.ServiceProxy('DobotServer/SetEndEffectorSuctionCup', SetEndEffectorSuctionCup)
        get_pose = rospy.ServiceProxy('DobotServer/GetPose', GetPose)
        set_home = rospy.ServiceProxy('DobotServer/SetHOMEParams', SetHOMEParams)
        publisher = rospy.Publisher('geometry_pose', Pose, queue_size=10)
        # subscriber = rospy.Subscriber
        
        pose = get_pose()
        home = set_home(0, 0, 0, 0, False)
        ini_x = pose.x
        ini_y = pose.y
        ini_z = pose.z
        
        print('a')
        # Valores arbitrarios para delimitar a area de atuacão do robo
        while True:
            print('The actual pose is')
            print('X: ')
            print(pose.x)
            print('Y: ')
            print(pose.y)
            print('Z: ')
            print(pose.z)
            print()

            print("Say where the object is positioned in the table")
            x = int(input("X axis: "))
            y = int(input("Y axis: "))
            z = int(input("Z axis: "))
            if 0 <= x > 200:
                print('X value is not acceptable')
            elif -100 < y > 100:
                print('Y value is not acceptable')
            elif -100 < z > 100:
                print('Z value is not acceptable')
            else:
                break

        freq = rospy.Rate(10)

        while not rospy.is_shutdown():
            msg = Pose()
            resp = suction_srv(1, 0, False)

            print('Move ate o ponto para pegar o objeto')
            # dist = 1
            while (pose.x != x and pose.y != y and pose.z != z) or not rospy.is_shutdown():
                msg.position.x = x
                msg.position.y = y
                msg.position.z = z
                publisher.publish(msg)
                # dist = ((x - pose.x) + (y - pose.y) + (z - pose.z))/3

                

            print('Suga')
            resp = suction_srv(1, 254, False)

            # dist = 1
            print('Move ate o ponto inicial')
            while (pose.x != ini_x and pose.y != ini_y and pose.z != ini_z) or not rospy.is_shutdown():
                msg.position.x = ini_x
                msg.position.y = ini_y
                msg.position.z = ini_z
                publisher.publish(msg)
                dist = ((x - pose.x) + (y - pose.y) + (z - pose.z))/3

            print('Para de sugar')
            resp = suction_srv(1, 0, False)

            freq.sleep()

            break
\end{lstlisting}

O código acima foi desenvolvido com base nos códigos desenvolvidos anteriormente e testado no manipulador até que consiga o movimenta-lo.