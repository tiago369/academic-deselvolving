\chapter{Introdução}
\label{chap:intro}




%--------- NEW SECTION ----------------------
\section{Objetivos}
\label{sec:obj}
Os objetivos são realizar a programação de um manipulador robótico especializado, em conjunto com visão computacional para poder realizar a automação de tarefas.
\label{sec:obj}

\subsection{Objetivos Específicos}
\label{ssec:objesp}
Os objetivos específicos deste projeto são:
\begin{itemize}
      \item Controlar um manipulador utilizando o ROS
      \item Identificar Objetos utilizando visão computacional
      \item Integrar o manipulador com a visão computacional
  \end{itemize}

%\subsubsection*{Objetivos específicos principais}
%\label{sssec:obj-principais}


%--------- NEW SECTION ----------------------
\section{Justificativa}
\label{sec:justi}

Justificativa

%--------- NEW SECTION ----------------------
\section{Organização do documento}
\label{section:organizacao}

Este documento apresenta $3$ capítulos e está estruturado da seguinte forma:

\begin{itemize}

  \item \textbf{Capítulo \ref{chap:intro} - Introdução}: Contextualiza o âmbito, no qual a pesquisa proposta está inserida. Apresenta, portanto, a definição do problema, objetivos e justificativas da pesquisa e como este \thetypeworkthree está estruturado;
  \item \textbf{Capítulo \ref{chap:08-02-22} - Dia: XX/XX/XX}: Explicita com toda a base teorica como o trabalho foi desenvolvido diariamente, o processo para a sua obtenção.

\end{itemize}
